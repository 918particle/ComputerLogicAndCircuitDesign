\documentclass{article}
\usepackage{graphicx}
\usepackage[margin=1.5cm]{geometry}
\usepackage{amsmath}
\usepackage{fancyvrb}
\usepackage{url}

\begin{document}

\title{Synopsis - Week 4: Trace Analyzer, Pattern Generator, and Stepping}
\author{Prof. Jordan C. Hanson}

\maketitle

\section{Introductions to Connect and Disconnect to PYNQ-Z1 System-on-a-Chip (SoC)}

\begin{enumerate}
\item Connect to the PYNQ-Z1 board in the usual fashion, ensuring that the system is logged out before starting a new kernel.
\item If troubleshooting is needed, consult lab partners or the professor.
\end{enumerate}

\section{Trace Analyzer and Pattern Generator}

\begin{enumerate}
\item Navigate to the \verb+logictools+ directory.
\item Run the script entitled \verb+pattern generator and trace analyzer+, and carefully follow instructions.  Notice the notation in the \verb+up_counter+ object for the timing diagram.
\item This notation is called waveJSON format.  It can also be used to display logic functions and gates.
\item \textbf{Modify} the \verb+up_counter+ structure to change from a \textit{3-bit counter} to a \textit{4-bit counter.}  Use the display function to verify the timing diagram.
\end{enumerate}

\section{Stepping}

\begin{enumerate}
\item Navigate to the \verb+single_stepping_generators+ notebook in \verb+logictools+.
\item Carefully follow instructions and execute each cell in order.
\item The \verb+logictools+ stepping function allows the firmware to (a) execute code as a digital firmware interface to the trace analyzer and pattern generator, and then (b) reprogram to form the boolean digital functions in the firmware and evaluate them, and (c) revert back.
\item Do you observe the correct output?
\end{enumerate}

\section{Logging Out}

Do not forget to log out of the system via the main page, 

\end{document}
