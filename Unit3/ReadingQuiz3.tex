\documentclass{article}
\usepackage{graphicx}
\usepackage[margin=1.5cm]{geometry}
\usepackage{amsmath}

\begin{document}

\title{Tuesday Reading Assessment: Chapters 6, 7}
\author{Prof. Jordan C. Hanson}

\maketitle

\section{Functions of Combinational Logic: Multiplexer}

\begin{enumerate}
\item (a) Draw a diagram of the gate logic describing the 4-to-1 multiplexer, using basic AND, OR, and inverter gates. The design should include the correct number of data-select lines, four inputs, and one output.  (b) Suppose that the data lines are called $D_i$, the data select lines are $S_i$, and the output line is $D_{out}$.  If the $S_i$ represent binary counting between 0-3, with 1 $\mu$s per clock step, how often can one send a bit of data to $D_{out}$ using the $D_0$ line? (c) How long would it take to send four bits of data, one on each $D_i$? \\ \vspace{4cm}
\end{enumerate}

\section{Functions of Combinational Logic: Demultiplexer}

\begin{enumerate}
\item (a) Draw a diagram of the gate logic describing the 1-to-4 demultiplexer, using basic AND, OR, and inverter gates. The design should include the correct number of data-select lines, one input, and four outputs.  (b) Suppose that the data input line is called $D_{in}$, the data output lines are $D_i$ the data select lines are $S_i$.  The $S_i$ represent binary counting between 0-3, with 1 $\mu$s per clock step.  Create a timing diagram showing the behavior of $D_{in}$, $S_i$, and $D_i$.  \\ \vspace{4cm}
\end{enumerate}

\end{document}
