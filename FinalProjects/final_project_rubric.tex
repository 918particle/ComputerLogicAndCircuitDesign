\title{Rubric for the Final Project, PHYS306/COSC330}
\author{Dr. Jordan Hanson - Whittier College Dept. of Physics and Astronomy}
\date{\today}
\documentclass[10pt]{article}
\usepackage[a4paper, total={18cm, 27cm}]{geometry}
\usepackage{outlines}
\usepackage[sfdefault]{FiraSans}
\usepackage{hyperref}

\begin{document}
\maketitle

\noindent
\textit{\textbf{Requirements}: 1. Propose an experiment to the professor by submitting a detailed project proposal.  2. Build the experiment and collect the data.  3. Create a 10-minute presentation on your results.  4. Give the presentation in the final week of class.} \\
\begin{itemize}
\item\textbf{ Project proposal}: The project proposal should be a 1-2 page description of the planned experiment.  It should include a diagram of the experiment, and include details about how the setup may be used to collect the appropriate data.  The proposal should be submitted on behalf of the group.  \textit{Group size limited to 2-3.  Data should be collected and shared remotely if possible.}
\begin{itemize}
\item Due date: April 8th, 2020.  Email project proposal to jhanson2@whittier.edu
\item One to two pages, with diagram and parts list
\end{itemize}
\item \textbf{Experiment}: The \textbf{applied logic} exercises proposed in the text are a good start for ideas.  The experiment should be a device or setup that is cheap, safe, and easy to build.  The experiment may be focused on a topic covered this semester, but it is not limited to that. It must either fall into the \textit{firmware}, \textit{hardware/firmware interface}, or DSP category.  Email jhanson2@whittier.edu with any questions about your project category.
\begin{itemize}
\item Data should be collected by April 28th, 2020
\item Presentation should begin by this date
\end{itemize}
\item \textbf{Presentation}: The presentation should be 7-10 slides and include an introduction of the concept.  Next, it should include and explanation of the setup (including a diagram).  Third, it should include the data, represented clearly and with correct units.  The statistical errors, and propagated errors should be quantified.  The hypothesis should be either confirmed or rejected.
\begin{itemize}
\item Presentations will be given via Zoom in front of the class
\item April 30th and May 5th
\end{itemize}
\item \textbf{Speaking}: When the group gives the presentation, the each member of the group should give at least part of the presentation.  Which parts and how much of the presentation is left to the group to decide.
\end{itemize}
\textit{\textbf{Example outline of the presentation}:}
\begin{outline}[enumerate]
\1 Slide 1: \textit{Implementation of Digital Comparators to Improve Image Smoothing} - by Jordan C. Hanson
\1 Slide 2: Introduction: ``Comparators are digital devices that ...can aide in image processing...''
\1 Slide 3: ``The following python code was run on the xyz machine ... the data showed improvement...''
\1 Slide 4: \textit{Tables of data for the...'' The data demonstrate the following effects, and include statistical errors...}
\1 Slide 5: ``The predicted image smoothing correlation coefficients are compared with the new use of comparators on the right...''
\1 Slide 6: ``In conclusion, the predicted coefficients were measured and the system appears to have been successful.''
\end{outline}
\textbf{Grading}: 30\% of the grade will be assigned based on \textit{attention to detail} in the project proposal.  What parts will you need?  What needs to be built?  Is this feasible?  Another 50\% will be assigned based on the execution of the experiment.  Are we allowing any unnecessary errors?  Are there any ways we can be more precise?  Finally, 20\% of the grade will be assigned on \textit{how clearly you related the findings to the class.}  Are you plotting or listing the data in such a way that other people can understand it?  Are there unit errors?  \textbf{\textit{Can people read your graphs and tables?} Finally, a bonus point will be awarded if the group makes use of the Zoom technology to enhance the presentation.  Examples include audio/visual effects, and file sharing or screen-sharing.}
\end{document}
