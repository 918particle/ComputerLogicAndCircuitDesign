\title{Syllabus for Computer Logic and Circuit Design: PHYS306/COSC330}
\author{Dr. Jordan Hanson - Whittier College Dept. of Physics and Astronomy}
\date{\today}
\documentclass[10pt]{article}
\usepackage[a4paper, total={18cm, 27cm}]{geometry}
\usepackage{outlines}
\usepackage[sfdefault]{FiraSans}
\usepackage{hyperref}
\begin{document}
\maketitle

\begin{abstract}
In this inaugural course on computer logic and digital circuit design will be divided into two parts: first, a \textbf{theoretical} mathematical basis for digital circuits will be introduced, followed by a tactile introduction to \textbf{experimentation} with digital circuits, concluding with \textit{analog to digital conversion}.  Concluding with analogue to digital conversion will form a bridge to \textit{digital signal processing} (DSP) - a course to be taught in Fall 2018.  The theoretical portion of the course will cover binary numerical representation, boolean algebra, truth tables, and combinatorial logic, culminating with the concept of tasks and functions represented by arrays of logic gates.  In the experimental portion of the course, the theory will be applied understanding digital circuit elements such as logic gate arrays, flip-flops, counters, memory elements (RAM), and analog to digital conversion.  The course will conclude (time permitting) with an introduction to microcontrollers. 
\end{abstract}
\noindent
\textit{\textbf{Pre-requisites}: PHYS180 and COSC120.  (This is the inaugural semester of a course which is expected to evolve over time.  The pre-requisites may be waved at the discretion of the instructor to admit a broad initial group of students).} \\
\textit{\textbf{Course credits, Liberal Arts Categorization}: 3 Credits, None} \\
\textit{\textbf{Regular course hours}: Tuesday and Thursday from 13:30 - 14:50 in SLC 104} \\
\textit{\textbf{Instructor contact information}: jhanson2@whittier.edu, tel. 562.907.5130} \\
\textit{\textbf{Office hours}: Monday 8:00-10:00 in SLC 212} \\
\textit{\textbf{Attendance/Absence}: Students needing to reschedule midterms and exams should notify the professor a reasonable time beforehand. Further attendance issues are left to the discretion of the instructor}.\\ 
\textit{\textbf{Late work policy}: Late work will not be accepted.} \\
\textit{\textbf{Text}: Learning the Art of Electronics by Thomas C. Hayes.  This is a laboratory book derived from text The Art of Electronics, 3rd Edition by Horowitz and Hill.  The text is a beautiful and vastly comprehensive desk reference for electronics design.  The laboratory book is recommended for purchase, but the text is not required.  Please inform the instructor if you cannot purchase the laboratory book, and arrangements will be made.} \\
\textit{\textbf{Grading}: There is no final exam for this course.  There will be two midterms, each worth 15\% of the final grade.  There will be two group presentations, each worth 15\% of the grade (see below).  Finally, the remaining 40\% of the grade will be homework comprised of reading quizzes, exercises, and selected simple circuit projects.} \\
\textit{\textbf{Grade Settings}: $<60\%$ = F, $>60\%,\leq 70\%$ = D, $>70\%,\leq80\%$ = C, $>80\%,\leq 90\%$ = B, $<90\%,\leq 100\%$ = A.  Pluses and minuses: 0-3\% minus, 3\%-6\% straight, 6\%-10\% plus (e.g. 79\% = C+, 91\% = A-)} \\
\textit{\textbf{Homework Sets}: Due weekly on Tuesdays.  Initial problem sets will focus on theoretical exercises from laboratory book and lecture notes.  As the semester progresses, the homework will shift to circuit projects, and submission of code or algorithms.} \\
\textit{\textbf{Bonus Essay}: Students may submit an essay on the history of scientific developments covered in the course, due at the end of the semester.  The essay must address scientific arguments and results, must include library references, and must have at least 10 pages.  The grade of this paper will replace the lowest midterm score if submitted.} \\
\textit{\textbf{ADA Statement on Disability Services}: The Americans with Disabilities Act (ADA) is a federal anti-discrimination statute that provides comprehensive civil rights protection for persons with disabilities. Among other things, this legislation requires that all students with disabilities be guaranteed a learning environment that provides for reasonable accommodation of their disabilities. If you believe you have a disability requiring an accommodation, please contact Disability Services: disabilityservices@whittier.edu, tel. 562.907.4825.} \\
\textit{\textbf{Academic Honesty Policy}: \url{http://www.whittier.edu/academics/academichonesty}} \\
\textit{\textbf{Course Objectives}:}
\begin{itemize}
\item To master binary logic and the application of binary logic to circuits
\item To train in the art of transforming a diagram into a working prototype
\item To develop confidence in electronics design and experimentation
\end{itemize}
\clearpage
\textit{\textbf{Course Outline}:}
\begin{outline}[enumerate]
\1 \textbf{Part 1: Introduction to circuits, and digital logic concepts - Weeks 1-6}
\2 Unit 1: Digital Mathematics and Logic
\3 Introduction to digital concepts
\3 \textit{Go!  Build a superheterodyne AM radio from a pile of parts and instructions}
\4 \textbf{How transistors work}
\4 Soldering, resistors, capacitors, and DC power (batteries)
\4 Kirchhoff's rules, Ohm's law
\3 Decimal, binary and hexidecimal number systems
\3 Numerical operations and codes
\2 Unit 2: Theoretical logic gates and gate operations
\3 Logic gates of various types, truth tables
\3 Programmable logic
\3 Boolean algebra and logic simplification
\3 Karnaugh maps
\2 Unit 3: Combinatorial logic, and functions of combinatorial logic 
\3 Basic combinatorial circuits
\3 Universal gates
\3 Time-dependent waveform analysis
\3 Adders, comparators, decoder/encoder, multiplexers
\1 \textbf{First midterm, and first set of group presentations}
\2 Midterm 1
\3 Conceptual understanding of AM radio functionality
\3 Numerical problems regarding binary numbers
\3 Logic problems regarding gates and their truth tables and Karnaugh maps
\2 Group presentation 1 - Design and test a circuit involving a binary adder
\3 Stage 1: propose a simple circuit or digital project, submit by \textit{Week 4}
\3 Stage 2: submit design, either copied from book, internet, or a custom design by \textit{Week 4}
\3 Stage 3: build and test project in groups \textit{as determined by group}
\3 Stage 4: present results, followed by group discussion and questions \textit{during Week 6}
\1 \textbf{Part 2: Experimentation with digital circuits - Weeks 7-13}
\2 Logic Gates laboratory - Learning AoE Unit 14L
\2 Flip-flops laboratory - Learning AoE Unit 15L
\2 Counters laboratory - Learning AoE Unit 16L
\2 Random Access Memory (RAM) laboratory - Learning AoE Unit 17L
\2 Analog to digital conversion laboratory - Learning AoE Unit 18L
\1 \textbf{Second midterm, and second set of group presentations, end of Week 13}
\2 Midterm 2
\3 Design of NAND gates (exam format)
\3 Design of flip-flops (exam format)
\3 Concept of analog to digital converion
\3 Debug a counter (lab format)
\2 Group presentation 2 - Design and test a digital circuit of group's choice
\3 Stage 1: propose a simple circuit or digital project, submit by \textit{Week 9}
\3 Stage 2: submit design, either copied from book, internet, or a custom design by \textit{Week 9}
\3 Stage 3: build and test project in groups \textit{as determined by group}
\3 Stage 4: present results, followed by group discussion and questions \textit{during Week 13}
\end{outline}
\end{document}
