\title{Homework 1 Solutions for Computer Logic and Circuit Design: PHYS306/COSC330}
\author{Dr. Jordan Hanson - Whittier College Dept. of Physics and Astronomy}
\date{\today}
\documentclass[10pt]{article}
\usepackage[a4paper, total={18cm, 27cm}]{geometry}
\usepackage{outlines}
\usepackage[sfdefault]{FiraSans}
\usepackage{hyperref}
\begin{document}
\maketitle

\section{1.1}

\begin{itemize}
\item Digital quantities don't suffer from noise or bandwidth issues in the same way as analog.  Digital quantities can be saved and replicated more easily.  Digital quantities have certain downsides as well.
\item Any scalar quantity that is physically observable: windspeed, current, voltage.
\end{itemize}

\section{1.2}

\begin{itemize}
\item Define the sequence of bits represented by each of the following levels: replace HIGH with 1's and LOW with 0's.
\item List the sequence of levels...opposite of previous problem.
\item Rise time: $\approx 1\mu$s, same as fall time.  Pulse width: $\approx 2\mu$s.  Amplitude: 10V.
\item Period of pulse sequence: 4 $\mu$s
\item Frequency: inverse of period, so 0.25 kHz or 250 Hz.
\item Periodic
\item The duty cycle is the pulse width divided by the period, so $(2/4) \mu$s or 50 percent.
\item 10101110
\item Serial transfer time: 8 $\mu$s.  Parallel: 1 $\mu$s.
\end{itemize}

\section{1.3}

\begin{itemize}
\item AND gate
\item AND gate
\item OR gate
\item Other answers were possible, but we've encountered AND, OR, and NOT in this chapter
\end{itemize}

\section{1.4}

\begin{itemize}
\item Adder, multiplier, multiplexer, comparator
\item $N = \Delta t \Delta f = 100~ms~\times~10~kHz~=1000$.  \textbf{Remember that the inverse of a millisecond is a kHz.}
\item 01010000
\end{itemize}

\section{1.5}

\begin{itemize}
\item LSI
\item DIP - dual inline package (the IC audio amp in the radios is a SIP, single inline package). SMT are surface mount parts, which are much smaller and don't require through-holes.  They require special machines to assemble.
\item Use the special notches to find pin 1.  From pin 1, the numbers proceed counter-clockwise.
\end{itemize}

\end{document}
