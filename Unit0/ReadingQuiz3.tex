\documentclass{article}
\usepackage{graphicx}
\usepackage[margin=1.5cm]{geometry}
\usepackage{amsmath}

\begin{document}

\title{Thursday Reading Assessment: Chapter 2-8, 2-10 through 2-12}
\author{Prof. Jordan C. Hanson}

\maketitle

\section{Hexadecimal, BCD, and Gray Codes}

\begin{enumerate}
\item Let $x_1 = F7_{16}$, and $x_2 = 9D_{16}$.
\begin{enumerate}
\item What is $x_1$ in binary? \\ \vspace{0.5cm}
\item What is $x_2$ in decimal? \\ \vspace{0.5cm}
\end{enumerate}
\item Imagine a stream of binary digits coming through a serial line.  They are: $1001 ... 0001 ... 0001$.  If we know the bitstream is in binary coded decimal (BCD), what is the code being sent? \\ \vspace{0.5cm}
\item Which of the following is an \textit{invalid} BCD code?
\begin{itemize}
\item A: 0001
\item B: 0101
\item C: 1000
\item D: 1010
\end{itemize}
\item What property of the four-bit gray code in Fig. \ref{fig:gray} distinguishes it from straight binary counting?
\begin{figure}[hb]
\centering
\includegraphics[width=0.5\textwidth,trim=0cm 3cm 0cm 0cm,clip=true]{figures/gray.pdf}
\caption{\label{fig:gray} A four-bit binary gray code.}
\end{figure}
\item Can you show how you would convert 1010 (gray code) to binary? \\ \vspace{0.5cm}
\item Observe Fig. \ref{fig:par} below, depicting the 4-bit BCD code.  Observe how the parity bit causes \textit{even} parity (even number of 1's), or \textit{odd} parity (odd number of 1's).  Circle all the following 4-bit BCD code words below that have a \textit{single-bit} error, assuming the parity bit is even:
\begin{itemize}
\item 100110010
\item 011101010
\item 10111111010001010
\end{itemize}
\begin{figure}[ht]
\centering
\includegraphics[width=0.5\textwidth]{figures/parity.pdf}
\caption{\label{fig:par} Even and odd parity bits.}
\end{figure}
\end{enumerate}

\end{document}
