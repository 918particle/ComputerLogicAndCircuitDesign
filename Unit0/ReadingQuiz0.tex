\documentclass{article}
\usepackage{graphicx}
\usepackage[margin=1.5cm]{geometry}
\usepackage{amsmath}

\begin{document}

\title{Thursday  Reading Assessment: Chapter 1}
\author{Prof. Jordan C. Hanson}

\maketitle

\section{Memory Bank}

\begin{enumerate}
\item Combining resistors \textit{in series}: $R_{\rm tot} = R_1 + R_2$
\item Combining resistors \textit{in parallel}: $R_{\rm tot}^{-1} = R_1^{-1} + R_2^{-1}$
\end{enumerate}

\section{Digital and Analog Concepts, Unit Conversions}

\begin{enumerate}
\item Perform the following unit conversions:
\begin{itemize}
\item Convert 0.24 V to mV. \\ \vspace{0.5cm}
\item Convert 0.035 A to mA. \\ \vspace{0.5cm}
\item If two 1200 $\Omega$ resistors are connected \textit{in series}, what is the total resistance in k$\Omega$? \\ \vspace{0.5cm}
\item If two 1200 $\Omega$ resistors are connected \textit{in parallel}, what is the total resistance in k$\Omega$? \\ \vspace{0.5cm}
\item If a digital pulse has a \textit{rise time} of 0.01 $\mu$s, what is the rise time in ns? \\ \vspace{0.5cm}
\item If a system can send a digital pulse with a minimum \textit{pulse width} of 300 ns, how many pulses per second can it send?  That is, what is the maximum pulse rate? \\ \vspace{0.5cm}
\end{itemize}
\item Suppose a system is passing an analog voltage signal down a wire, with \textit{amplitude} $a$, and \textit{frequency} $f_1$:
\begin{equation}
v(t) = a\cos(2\pi f_1 t) + v_0
\end{equation}
(a) \textbf{Digitizing} the signal means breaking the amplitude variable into discrete components.  If we can measure 256 discrete voltages between 0 and 5V, and $a = 2.5$ V, $v_0 = 2.5$ V, what is the smallest change in voltage we can measure? (b) \textbf{Sampling} the signal means breaking the time variable into discrete components.  Suppose $f_1 = 60$ Hz.  This means the \textit{period} is $1/60$ seconds.  If we sample $v(t)$ at a rate of 1 kHz, how many samples per period will we collect?
\end{enumerate}

\end{document}
